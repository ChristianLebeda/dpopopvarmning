\problemname{Pariserhjulet}
När $N$ st IOI lag var ute och åkte buss på den Azerbajanska steppen stötte de på en nöjespark med ett jättestort pariserhjul! Facinerade av att det kunde finnas pariserhjul utanför Paris ville de förståss alla åka. Pariserhjulet har $M$ vagnar som rymmer exakt ett IOI lag vardera och hjulet snurrar med en hastighet $1/M$ varv per minut. Lagen har nu bildat en lång kö framför den spektakulära attraktionen. Lag $i$ vill åka $T_i$ varv i pariserhjulet. Om en ledig vagn kommer ner till påstigningsplatsen eller om en vagn kommer ner samtidigt som något lag stiger av kommer laget längst fram i kön stiga på vagnen och sedan åka alla sina varv utan att stiga av emellan.
En uttråkad busschaufför frågar dig hur länge han måste vänta tills att alla lag har åkt färdigt

\section*{Indata}
Den första raden innehåller heltalen $N$ och $M$ separerade med blanksteg,
antalet lag och antalet vagnar i pariserhjulet.
Den andra raden innehåller $N$ heltal $T_1 ... T_N$ separerade med blanksteg,
där $T_i$ är antalet varv lag nummer $i$ vill åka. Lagen är ordnade efter
köplats, där $T_1$ är det första laget i kön.

\section*{Utdata}
Skriv ut en rad med ett heltal, antalet minuter det kommer ta för alla lag att åka.

\section*{Begränsningar}
Din lösning kommer att testas på en uppsättning testgrupper, var och en värd en viss poäng.
Varje testgrupp innehåller flera testfall.
För att få poäng för en testgrupp måste du klara alla testfallen i gruppen.
Din slutgiltiga poäng på problemet kommer att vara den maximala poängen av en enda inskickning.

\noindent
\begin{tabular}{| l | l | l |}
\hline
Grupp & Poäng & Gränser \\ \hline
1     & 20    & $1 \le N, M, T_i \le 100$ \\ \hline
2     & 30    & $1 \le N, M, T_i \le 1000$ \\ \hline
3     & 25    & $1 \le N, M \le 1000, 1 \le T_i \le 10^9$ \\ \hline
4     & 25    & $1 \le N, M \le 200,000, 1 \le T_i \le 10^9$ \\ \hline
\end{tabular}
