\problemname{Tebryggning}
Esmeralda ska brygga massa te till $N$ stycken programmeringsolympiadsdeltagare.
Hon har $K$ påsar te och alla påsar är av olika sorter.
Påse $i$ har te för $x_i$ personer.
Esmeralda tänker använda bryggkannor som har plats för te till 10 personer.
Eftersom alla påsar är av olika sort
går det ej att använda flera påsar i samma kanna.
Dock kan samma påse användas till flera kannor.

Hur många kannor måste Esmeralda brygga?

Det är garanterat att det är möjligt att brygga te till alla

\section*{Indata}
På den första raden står två heltal $1 \le N \le 100$ och $1 \le K \le 10$
 -- antalet programmeringsolympiadsdeltagare och antalet tepåsar Esmeralda har.
På den andra raden står $K$ heltal $1 \le x_1, x_2, \dots, x_N \le 100$,
hur många personer varje påse räcker till.

\section*{Utdata}
Programmet ska skriva ut ett heltal: det minsta antalet tekannor Esmeralda måste brygga. 

\section*{Poängsättning}
För testfall värda $1$ poäng gäller att $K=1$.

\section*{Förklaring av exempel}
I exempel 1 väljer Esmeralda att brygga två kannor med första tepåsen 
och två kannor med tredje tepåsen. Det ger $20+17$ koppar te, vilket
räcker till de 36 deltagarna.

I exempel 2 är det optimala att brygga sex kannor med första tepåsen,
tre kannor med tredje tepåsen och två med den fjärde tepåsen.
Det ger $54+30+16$ koppar te, vilket räcker till de 100  deltagarna.