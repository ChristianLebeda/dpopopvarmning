\problemname{Planet X}
Året är 2109 och en grupp forskare har just upptäckt ``Planet X'', 
en tidigare okänd planet här i vårt egna solsystem,
bortom Plutos omloppsbana. Genast skickar forskargruppen ut
en sond för att göra mätningar, och kort därefter får de tillbaka mätdata.

Forskarna är specifikt intresserade av hur ytan på Planet X ser ut.
Vi representerar här ytan som ett $N \times M$ rutnät, där varje ruta
har en höjd mellan 0 och 9.

Ett mätinstrument på sonden har lyckats mäta den specifika höjden
på vissa, men inte alla, rutor. Utifrån den kemiska sammansättningen i ytan vet vi att det inte 
är särskilt brant på planeten: höjden mellan två
intilliggande rutor (rutor som delar en kant) aldrig kan skilja 
med mer än ett. 

Nu behöver forskarna din hjälp för att få ut så mycket information
från denna data som möjligt. Närmare bestämt vill de att du givet höjden
på några av rutorna hittar höjden på alla andra rutor som går att bestämma entydigt.

\section*{Indata}
På den första raden står två heltal $1 \le N,M \le 10$, 
höjden på rutnätet och bredden på rutnätet respektive.
Därefter följer $N$ rader med $M$ tecken på varje.
Det $j:te$ tecknet på rad $i$ är en \texttt{.} ifall
inget värde för denna ruta finns, och är en siffra mellan
0 och 9 som motsvarar höjden på rutan annars.

\section*{Utdata}
Programmet ska skriva ut $N$ rader med $M$ tecken på varje:
rutnätet som det ser ut efter att
korrekta höjder är ifyllda på alla rutor där höjden går att bestämma.

\section*{Poängsättning}
För testfall värda $20$ poäng gäller att $N=1$. \\
För testfall värda $20$ poäng gäller att $N,M \leq 3$. \\
För de resterande testfallen (värda $60$ poäng) gäller att $1\leq N,M \leq 10$.
