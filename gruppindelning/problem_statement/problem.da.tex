\problemname{Gruppeinddeling}

Under en skoleudflugt skal eleverne deles ind i forskellige grupper.
Naturligvis vil eleverne være i samme gruppe som deres venner.
Skriv et program, der givet hver elevs navn og venskabsforhold udregner det maksimale antal grupper, som eleverne kan inddeles i (under antagelse af, at eleverne får det, som de vil have det).

\section*{Indlæsning}

På første linje står et heltal: antallet af elever, som skal på udflugt ($2 \le n \le 100$).
Derefter følger $n$ linjer, hver bestående af navnet på en enkelt elev.
Hvert navn består af mellem $1$ og $20$ tegn og indeholder kun bogstaverne \texttt{A-Z}.
Alle elever har forskellige navne.

Derefter følger en linje med et enkelt heltal: antallet af vennepar ($1 \le m \le 4950$).
Endeligt følger $m$ linjer med venneparrene.
For hvert par angives to navne på samme linje, adskilte af et enkelt mellemrum.

\section*{Udskrift}
Programmet skal udskrive en linje med et heltal: det maksimale antal grupper, som eleverne kan inddeles i.
